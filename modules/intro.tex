% == FILE: modules/intro.tex ==
% Module: Introduction to RTL-to-GDSII Flow

\subsection{Overview}
The RTL-to-GDSII (Register Transfer Level to Graphic Data Stream Information Interchange) flow is the complete process of transforming a high-level hardware description into a physical layout ready for semiconductor fabrication. This workshop provides hands-on experience with open-source tools that enable this transformation.

\subsection{What is RTL-to-GDSII?}
The RTL-to-GDSII flow encompasses several critical stages:

\begin{enumerate}
  \item \textbf{RTL Design} -- Hardware description using languages like Verilog or VHDL
  \item \textbf{Verification} -- Functional simulation and validation
  \item \textbf{Synthesis} -- Conversion of RTL to gate-level netlist
  \item \textbf{Floor Planning} -- Arrangement of major functional blocks
  \item \textbf{Placement} -- Positioning of standard cells
  \item \textbf{Clock Tree Synthesis} -- Distribution of clock signals
  \item \textbf{Routing} -- Creation of metal interconnections
  \item \textbf{Verification} -- Physical verification (DRC, LVS, timing)
  \item \textbf{GDSII Generation} -- Final layout for fabrication
\end{enumerate}

\subsection{Why Open-Source Tools?}
Open-source EDA (Electronic Design Automation) tools have revolutionized hardware design by:

\begin{itemize}
  \item \textbf{Democratizing Access} -- Making chip design accessible to students, researchers, and small companies
  \item \textbf{Transparency} -- Allowing users to understand and modify tool behavior
  \item \textbf{Cost Reduction} -- Eliminating expensive licensing fees
  \item \textbf{Community Support} -- Fostering collaboration and rapid innovation
  \item \textbf{Educational Value} -- Providing learning opportunities without financial barriers
\end{itemize}

\subsection{Workshop Objectives}
By the end of this workshop, participants will be able to:

\begin{itemize}
  \item Set up and configure a complete open-source RTL-to-GDSII toolchain
  \item Design and verify digital circuits using Verilog
  \item Perform logic synthesis using Yosys
  \item Conduct physical design using OpenROAD
  \item Verify designs using open-source verification tools
  \item Generate GDSII files ready for fabrication
  \item Automate design flows using TCL scripting
\end{itemize}

\subsection{Target Audience}
This workshop is designed for:

\begin{itemize}
  \item Digital design engineers transitioning to open-source tools
  \item Graduate students studying VLSI design
  \item Researchers exploring hardware design automation
  \item Educators preparing curriculum for chip design courses
  \item Hobbyists interested in semiconductor design
\end{itemize}

\subsection{Prerequisites}
Participants should have:

\begin{itemize}
  \item Basic understanding of digital logic design
  \item Familiarity with Verilog or VHDL
  \item Linux command-line experience (basic level)
  \item Access to a Linux system (native or WSL)
  \item At least 20GB of free disk space for tools and libraries
\end{itemize}

\subsection{Workshop Structure}
The workshop follows a progressive learning path:

\begin{enumerate}
  \item Environment setup and foundational tools
  \item Scripting and automation techniques
  \item Front-end design (HLS, RTL, verification)
  \item Synthesis and optimization
  \item Physical design and implementation
  \item Verification and sign-off
\end{enumerate}

Each module includes theoretical background, practical examples, hands-on exercises, and real-world design considerations.
