% == FILE: modules/module_sta.tex ==
% Module: Static Timing Analysis using OpenSTA

\subsection{Overview}
This module introduces \textbf{OpenSTA}, an open-source tool for performing \textbf{Static Timing Analysis (STA)} on gate-level netlists.  
STA is used to verify that the design meets its timing constraints — setup, hold, and clock requirements — without the need for dynamic simulation.

\subsection{Objectives}
\begin{itemize}
  \item Install and configure OpenSTA.
  \item Perform setup and hold analysis on a synthesized netlist.
  \item Understand timing reports and interpret slack values.
\end{itemize}

\subsection{Installing OpenSTA}
OpenSTA requires basic development tools and libraries. On Ubuntu or WSL, install dependencies and build from source:
\begin{lstlisting}[language=bash]
# Clone the OpenSTA repository
git clone https://github.com/The-OpenROAD-Project/OpenSTA.git
cd OpenSTA
mkdir build && cd build

# Configure and compile
cmake ..
make
sudo make install
\end{lstlisting}

Verify the installation:
\begin{lstlisting}[language=bash]
sta
\end{lstlisting}
If the STA prompt appears, installation was successful.  
Use the `exit` command to quit.

\subsection{Files Required for Analysis}
OpenSTA requires the following four files:
\begin{itemize}
  \item \texttt{top.v} – Gate-level Verilog netlist.
  \item \texttt{top.sdc} – Timing constraint file.
  \item \texttt{toy.lib} – Technology library file (Liberty format).
  \item \texttt{test.tcl} – Tcl script to automate the analysis.
\end{itemize}

\subsection{Example Constraint File (top.sdc)}
The SDC file defines clocks and I/O delays:
\begin{lstlisting}[language=tcl]
create_clock -name CLK -period 1000 [get_ports clk]
set_input_delay 5 -clock CLK [get_ports a]
set_input_delay 5 -clock CLK [get_ports b]
set_output_delay 5 -clock CLK [get_ports out]
\end{lstlisting}

\subsection{Example TCL Script (test.tcl)}
The TCL script automates the OpenSTA workflow:
\begin{lstlisting}[language=tcl]
# Read library, design, and constraints
read_liberty toy.lib
read_verilog top.v
link_design top
read_sdc top.sdc
\end{lstlisting}

Run OpenSTA using:
\begin{lstlisting}[language=bash]
sta source test.tcl
\end{lstlisting}

\subsection{Timing Analysis Commands}
After loading the design, use these commands for timing verification:
\begin{lstlisting}[language=tcl]
# Maximum delay path (setup check)
report_checks -path_delay max -fields {slew delay time} -format full

# Minimum delay path (hold check)
report_checks -path_delay min -fields {slew delay time} -format full
\end{lstlisting}

You can also export results to a text file:
\begin{lstlisting}[language=tcl]
report_checks -path_delay max -format full > setup_report.txt
report_checks -path_delay min -format full > hold_report.txt
\end{lstlisting}

\subsection{Interpreting the Timing Report}
A typical setup analysis output includes:
\begin{itemize}
  \item \textbf{Startpoint:} Launch flip-flop (e.g., F1).
  \item \textbf{Endpoint:} Capture flip-flop (e.g., F2).
  \item \textbf{Clock period:} 1000 ps (from constraint file).
  \item \textbf{Data arrival time:} Sum of individual gate and wire delays.
  \item \textbf{Data required time:} Clock period minus setup time.
  \item \textbf{Slack:} Difference between required and arrival times.
\end{itemize}

A positive slack indicates no setup violation.  
If slack is negative, the circuit fails timing and must be optimized.

\subsection{Hold Analysis}
Hold check ensures data stability between flip-flops on the same clock edge:
\begin{itemize}
  \item \textbf{Startpoint:} F2 (launch flop).
  \item \textbf{Endpoint:} F3 (capture flop).
  \item \textbf{Hold time:} Extracted from the Liberty file.
  \item \textbf{Slack:} Data arrival time minus data required time.
\end{itemize}

Positive hold slack = no violation.

\subsection{Key Observations}
\begin{itemize}
  \item Setup and hold checks confirm the design’s timing health.
  \item All timing data (delays, setup/hold times) are extracted from the Liberty file.
  \item OpenSTA allows flexible reporting and TCL-based automation.
\end{itemize}

\subsection{Exercise}
\begin{enumerate}
  \item Install OpenSTA and verify the setup using the example project.
  \item Modify \texttt{top.sdc} to experiment with different clock periods and observe timing slack.
  \item Create separate reports for setup and hold checks.
  \item Identify critical paths from the reports.
\end{enumerate}

\subsection{Summary}
OpenSTA performs comprehensive static timing verification for gate-level designs.  
By analyzing maximum and minimum delay paths, it ensures that setup and hold constraints are satisfied before proceeding to placement, routing, and GDSII generation.

\subsection{Technology Library and Constraints}

This section extends the previous tutorial by explaining how delay calculations and static timing results are affected by the \textbf{technology library (Liberty file)} and \textbf{design constraints (SDC)}.  
We will again use \textbf{OpenSTA} for demonstration.

\subsubsection*{Objective}
\begin{itemize}
  \item Understand how NLDM (Nonlinear Delay Model) tables in the technology library determine gate delay.
  \item Study the impact of constraints such as input transition, output load, and clock uncertainty on timing results.
  \item Perform practical experiments using OpenSTA to observe delay and slack variations.
\end{itemize}

\subsection{Nonlinear Delay Model (NLDM) Overview}
In Liberty (.lib) format, delay between input and output pins of a cell is modeled as a two-dimensional lookup table.  
The delay depends on:
\begin{itemize}
  \item \textbf{Input transition (slew)} – how fast the input signal changes.
  \item \textbf{Output load capacitance} – the load driven by the output pin.
\end{itemize}

For example, an inverter cell may have a timing arc defined as:

\begin{verbatim}
cell (INV) {
  pin (ZN) {
    direction : output;
    related_pin : "A";
    timing () {
      cell_rise (delay_template) {
        index_1 ("0.1, 100"); // input transition
        index_2 ("0.1, 100"); // output load
        values ( "184, 200", "190, 210" );
      }
    }
  }
}
\end{verbatim}

Here, the delay is tabulated as a function of input slew and output capacitance.  
For realistic libraries, the tables are larger with many interpolation points.

\subsection{Constraint Types in SDC}
Key constraint commands used in OpenSTA analysis:
\begin{itemize}
  \item \textbf{create\_clock} – Defines the primary clock and its period.
  \item \textbf{set\_input\_delay} / \textbf{set\_output\_delay} – Specifies external delay margins for I/O ports.
  \item \textbf{set\_input\_transition} – Defines the input signal slew rate.
  \item \textbf{set\_load} – Sets the capacitive load for output ports.
  \item \textbf{set\_clock\_uncertainty} – Models clock jitter or variation to make timing analysis more conservative.
\end{itemize}

\subsection{Example Experiment}
Consider a simple design \texttt{test.v} containing one inverter:
\begin{lstlisting}[language=verilog]
module top(input A, output out);
  INV i (.A(A), .ZN(out));
endmodule
\end{lstlisting}

And its corresponding constraint file \texttt{test.sdc}:
\begin{lstlisting}[language=tcl]
create_clock -name CLK -period 1000
set_input_delay 5 -clock CLK [get_ports A]
set_output_delay 5 -clock CLK [get_ports out]
set_input_transition 0.1 [get_ports A]
set_load 100 [get_ports out]
\end{lstlisting}

The input transition and load values correspond to the characterization points in the NLDM table of the toy library.

\subsection{Expected Delay Calculation}
Referring to the NLDM table for the inverter:
\begin{itemize}
  \item Input transition = 0.1 ps (row 1)
  \item Output load = 100 fF (column 2)
  \item Delay from input to output = \textbf{80 ps}
\end{itemize}

Hence, total arrival time at the output:
\[
T_{arrival} = T_{input\_delay} + T_{cell\_delay} = 5 + 80 = 85\text{ ps.}
\]
Required time (considering 1000 ps clock and 5 ps output delay):
\[
T_{required} = 1000 - 5 = 995\text{ ps.}
\]
Slack = \(995 - 85 = 910\) ps (positive, no violation).

\subsection{Effect of Changing Constraints}
\begin{itemize}
  \item Increasing input transition from 0.1 ps to 100 ps increases delay (e.g., from 80 ps → 200 ps).
  \item Reducing output load from 100 fF to 0.1 fF decreases delay (e.g., from 200 ps → 4 ps).
  \item Increasing input delay or output delay reduces available slack.
  \item Adding clock uncertainty (e.g., 100 ps) decreases required time by the same amount, reducing slack.
\end{itemize}

\subsection{Practical Observations}
\begin{itemize}
  \item Cell delay grows with higher input slew and load capacitance.
  \item Constraints directly influence setup and hold slack reported by OpenSTA.
  \item Designers can perform what-if experiments by varying constraint values to study timing sensitivity.
\end{itemize}

\subsection{Further Reading and References}
\begin{itemize}
  \item OpenSTA official documentation: \url{https://github.com/The-OpenROAD-Project/OpenSTA}
  \item Liberty File Format Reference: \url{https://openroad.readthedocs.io/en/latest/OpenSTA/Liberty.html}
  \item Synopsys Design Constraints (SDC) Reference: \url{https://www.synopsys.com/support/sdc.html}
  \item NPTEL Course: VLSI Design Flow – RTL to GDS (Week 8, Dr. Sneh Saurabh)
\end{itemize}

\subsection{Summary}
This experiment demonstrated how the delay model in a technology library and design constraints in the SDC file affect static timing analysis.  
By modifying constraints such as input transition, load, and clock uncertainty, one can observe predictable changes in delay and slack.  
A strong understanding of these relationships is essential for accurate timing closure in real-world chip design.
