% == FILE: modules/module_tcl.tex ==
% Module: Introduction to TCL (Tool Command Language)

\subsection{Overview}
This module introduces TCL (Tool Command Language), a powerful scripting language extensively used in Electronic Design Automation (EDA) tools and design flows. TCL enables automation, tool customization, and integration between multiple tools in the RTL-to-GDSII process.

\subsection{Purpose in VLSI Design}
TCL scripting allows designers to:
\begin{itemize}
  \item Automate repetitive commands in synthesis, place-and-route, and verification.
  \item Interface multiple EDA tools in a unified flow.
  \item Create parameterized design scripts for better reproducibility.
  \item Modify and control tool environments dynamically.
\end{itemize}

\subsection{Basic TCL Commands and Syntax}
TCL syntax is similar to many programming languages and includes constructs such as loops, conditionals, lists, and procedures.

\subsubsection*{Example 1: Basic List Operations}
\begin{lstlisting}[style=tclstyle]
set List {0 1 2 3 4 5 6}
set index -1
foreach elem $List {
  incr index
  puts "Index: $index"
  if {$elem % 2 == 0} {
    lset List $index [expr {-$elem}]
  }
  puts "Updated list: $List"
}
\end{lstlisting}
\textbf{Explanation:}
\begin{itemize}
  \item \textbf{set} – Defines a variable and assigns a value.
  \item \textbf{foreach} – Iterates over each element in a list.
  \item \textbf{incr} – Increments a numeric variable.
  \item \textbf{puts} – Prints to the terminal.
  \item \textbf{if} – Conditional check similar to other languages.
  \item \textbf{lset} – Updates an element of a list at a given index.
  \item \textbf{expr} – Evaluates an arithmetic expression.
\end{itemize}

\subsubsection*{Example 2: File Handling Commands}
\begin{lstlisting}[style=tclstyle]
set fp [open "input.txt" w+]
puts $fp "test"
close $fp

set fp [open "input.txt" r]
set file_data [read $fp]
puts $file_data
close $fp
\end{lstlisting}
\textbf{Explanation:}
\begin{itemize}
  \item \textbf{open} – Opens a file channel in a specific mode (read, write, append, etc.).
  \item \textbf{close} – Closes the file channel.
  \item \textbf{puts} – Writes data to a file or displays output.
  \item \textbf{read} – Reads file contents into a variable.
\end{itemize}

\subsubsection*{Example 3: Procedures and Return Statements}
\begin{lstlisting}[style=tclstyle]
proc printSumProduct {x y} {
  set sum [expr {$x + $y}]
  set prod [expr {$x * $y}]
  puts "Sum is : $sum"
  puts "Product is : $prod"
  return
  puts "This line will not be printed"
}

puts [printSumProduct 10 50]
\end{lstlisting}
\textbf{Explanation:}
\begin{itemize}
  \item \textbf{proc} – Defines a reusable procedure or function.
  \item \textbf{return} – Terminates execution and returns control to the caller.
\end{itemize}

\subsubsection*{Example 4: Executing System Commands}
\begin{lstlisting}[style=tclstyle]
puts [exec ls]
puts [exec pwd]
\end{lstlisting}
\textbf{Explanation:}
\begin{itemize}
  \item \textbf{exec} – Executes external system commands within a TCL script.
  \item In this example, the script lists directory contents and prints the current working directory.
\end{itemize}

\subsection{Running TCL Scripts}
TCL scripts are saved with a \\texttt{.tcl} extension and executed using:
\begin{lstlisting}[style=bashstyle]
tclsh script_name.tcl
\end{lstlisting}

\subsection{Practical Usage in VLSI Flow}
TCL is widely embedded in open-source and commercial tools such as Yosys, OpenROAD, Magic, and Synopsys tools. It is used to:
\begin{itemize}
  \item Configure design environments.
  \item Execute synthesis and timing analysis scripts.
  \item Generate reports automatically.
  \item Manage flow automation pipelines.
\end{itemize}

\subsection{Summary}
TCL scripting is a cornerstone of modern EDA automation. A clear understanding of its commands and structure is essential for customizing design flows and improving productivity in RTL-to-GDSII projects.
