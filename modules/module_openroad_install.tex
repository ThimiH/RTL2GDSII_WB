% == FILE: modules/module_openroad_install.tex ==
% Module: Installing and Preparing OpenROAD

\subsection{Overview}
This section explains the installation and setup of \textbf{OpenROAD}, an open-source integrated physical design tool. OpenROAD automates chip layout tasks — including floorplanning, placement, clock tree synthesis, and routing — transforming a synthesized Verilog netlist into a final routed layout.

\subsection{About OpenROAD}
OpenROAD (Open Rapid Object-oriented Automated Design) provides a complete RTL-to-GDSII physical design flow, handling:
\begin{itemize}
  \item Floorplanning and Power Planning
  \item Placement and Clock Tree Synthesis (CTS)
  \item Global and Detailed Routing
  \item Parasitic Extraction and Reporting
\end{itemize}

\subsection{Installation Steps}
Execute these commands in a Linux or WSL terminal:
\begin{lstlisting}[style=bashstyle]
# Clone repository
git clone --recursive \
    https://github.com/The-OpenROAD-Project/OpenROAD.git
cd OpenROAD

# Install dependencies
sudo ./etc/DependencyInstaller.sh

# Build OpenROAD
mkdir build
cd build
cmake ..
make -j$(nproc)
sudo make install
\end{lstlisting}

Installation may take about 1–2 hours. After completion, run:
\begin{lstlisting}[style=bashstyle]
openroad
\end{lstlisting}

\subsection{Included Libraries and Examples}
The OpenROAD repository includes preconfigured libraries:
\begin{itemize}
  \item \textbf{Nangate45}
  \item \textbf{ASAP7}
  \item \textbf{SkyWater130}
\end{itemize}

Each library folder provides example Verilog files, constraint files, and TCL scripts for step-by-step flow demonstrations.

\subsection{Next Steps}
Once installed, users can explore sample scripts in:
\begin{lstlisting}[style=bashstyle]
OpenROAD/test/
\end{lstlisting}
These include examples such as \texttt{gcd\_nangate45.tcl}, which will be used for chip planning and placement in the following module.
