% == FILE: modules/module_openroad_physical.tex ==
% Module: Physical Design Flow using OpenROAD

\subsection{Overview}
This section covers the complete physical design process using OpenROAD, from floorplanning to routing. We continue with our \textbf{GCD unit golden thread example}, taking the same design we:
\begin{itemize}
  \item Generated using HLS (Section 5)
  \item Manually designed at RTL (Section 6)
  \item Verified through simulation (Section 7)
  \item Synthesized to gates with Yosys (Section 8)
\end{itemize}

We now perform \textbf{place and route} to create a physical layout that can be fabricated. The example uses the \texttt{gcd\_nangate45.tcl} script with the Nangate45 library.

\textbf{Key Continuity:} This is the same GCD algorithm computing greatest common divisor — we're now translating it from abstract gates to concrete silicon geometry.

\subsection{Inputs Required}
\begin{itemize}
  \item Synthesized Verilog netlist (\texttt{gcd.v} or \texttt{gcd\_synth.v} from Yosys)
  \item Constraint file (\texttt{gcd\_nangate45.sdc}) specifying timing requirements
  \item Technology files:
    \begin{itemize}
      \item \texttt{Nangate45\_typ.lib} (timing library)
      \item \texttt{Nangate45\_stdcell.lef} (standard cell layout)
      \item \texttt{Nangate45\_tech.lef} (routing rules and metal stack)
    \end{itemize}
  \item Flow TCL script: \texttt{gcd\_nangate45.tcl}
\end{itemize}

\textbf{Important Configuration Step:}
\begin{itemize}
  \item If using the default \texttt{gcd\_nangate45.tcl} from OpenROAD examples, you must \textbf{edit the script} to point to YOUR synthesized netlist from Section 8.
  \item Locate the line: \texttt{read\_verilog <path>/gcd.v}
  \item Change it to: \texttt{read\_verilog ./gcd\_synth.v} (or the full path to your Yosys output)
  \item Similarly, ensure the SDC file path points to your constraints file.
\end{itemize}

\textbf{Note:} The GCD netlist contains the FSM states, datapath registers (a, b, result), subtractor logic, and comparators we designed earlier. OpenROAD will place these gates and route the connections between them.

\subsection{1. Floorplanning and Power Planning}
\begin{itemize}
  \item Define die/core area using bottom-left (LX, LY) and top-right (UX, UY) coordinates.
  \item Generate site rows for cell placement.
  \item Place IO pins randomly or by constraint.
  \item Add tap and end-cap cells to prevent latch-up and protect boundaries.
  \item Generate the Power Distribution Network (PDN) using:
  \begin{lstlisting}[style=tclstyle]
source flow_pdn.tcl
  \end{lstlisting}
  This builds VDD and VSS grids across metal layers.
\end{itemize}

\subsection{2. Placement}
\begin{itemize}
  \item Perform global placement with congestion estimation.
  \item Set signal and clock routing layers.
  \item Repair violations (slew, capacitance, fanout) and resize gates.
  \item Execute detailed placement to legalize cell positions.
\end{itemize}

The resulting DEF shows correctly aligned cells within placement rows. You can identify:
\begin{itemize}
  \item \textbf{State register flip-flops} for the FSM
  \item \textbf{Datapath registers} holding \texttt{a}, \texttt{b}, and \texttt{result}
  \item \textbf{Arithmetic logic} for the subtractors and comparators
  \item \textbf{Control multiplexers} selecting between operations
\end{itemize}

\subsection{3. Clock Tree Synthesis (CTS)}
\begin{itemize}
  \item Clone and repair clock inverters.
  \item Build the clock tree with buffer insertion and clustering:
  \begin{lstlisting}[style=tclstyle]
clock_tree_synthesis -root clk -buf_list {BUF_X2 BUF_X4}
  \end{lstlisting}
  \item Re-run detailed placement for clock buffers.
  \item Repair setup and hold violations after CTS.
\end{itemize}

\subsection{4. Routing}
\textbf{Global Routing:}
\begin{itemize}
  \item Generates route guides using global congestion analysis.
  \item Identifies congestion and antenna issues.
\end{itemize}

\textbf{Detailed Routing:}
\begin{itemize}
  \item Uses multiple threads for performance.
  \item Creates metal traces respecting DRC rules.
  \item Inserts filler cells and checks antenna violations.
  \item Outputs final \texttt{.def}, \texttt{.gds}, and \texttt{.spef} (parasitics) files.
\end{itemize}

\subsection{5. Parasitic Extraction and Reporting}
After routing, extract RC data:
\begin{lstlisting}[style=tclstyle]
extract_parasitics -rc_file nangate45.rc
report_timing
report_power
\end{lstlisting}
Generates timing and power summaries, worst slack reports, and congestion maps.

\subsection{Visualization and Checks}
\begin{itemize}
  \item Use the GUI command:
  \begin{lstlisting}[style=bashstyle]
openroad -gui -log gcd.log -script gcd_nangate45.tcl
  \end{lstlisting}
  \item Explore:
    \begin{itemize}
      \item Placement density (heatmap)
      \item Clock tree topology
      \item Routing congestion and power grid
    \end{itemize}
\end{itemize}

\subsection{Summary}
This module completes the physical design phase in the RTL-to-GDSII flow. You learned how OpenROAD automates floorplanning, placement, CTS, and routing, producing a verified and routed layout ready for DRC/LVS and GDS export.

\textbf{Workshop Golden Thread Complete:} We have now taken the GCD algorithm from:
\begin{enumerate}
  \item High-level C code (HLS synthesis)
  \item Manual RTL design (FSM + Datapath)
  \item Functional verification (Simulation)
  \item Gate-level netlist (Logic synthesis with optimization)
  \item Physical layout (Place and route)
\end{enumerate}

This unified example demonstrates the complete RTL-to-GDSII flow using a single, coherent design throughout all stages.
