% == FILE: modules/schedule.tex ==
% Module: Workshop Schedule

\subsection{Workshop Format}
This workshop is structured as a progressive, hands-on learning experience combining:
\begin{itemize}
  \item Theoretical presentations and tool demonstrations
  \item Guided practical exercises
  \item Independent design challenges
  \item Group discussions and troubleshooting sessions
\end{itemize}

\subsection{Recommended Schedule (5-Day Format)}

\subsubsection{Day 1: Foundation and Environment Setup}
\textbf{Morning Session (9:00 AM - 12:00 PM)}
\begin{itemize}
  \item Introduction to RTL-to-GDSII flow
  \item Toolchain overview and architecture
  \item Linux environment setup (WSL/Ubuntu)
  \item Installing essential tools and dependencies
  \item Unix command-line basics
\end{itemize}

\textbf{Afternoon Session (1:00 PM - 5:00 PM)}
\begin{itemize}
  \item TCL scripting fundamentals
  \item Writing automation scripts
  \item Tool invocation and flow control
  \item \textbf{Lab:} Create automated build scripts
\end{itemize}

\subsubsection{Day 2: Design Entry and Verification}
\textbf{Morning Session (9:00 AM - 12:00 PM)}
\begin{itemize}
  \item High-Level Synthesis with Bambu
  \item C-to-RTL conversion techniques
  \item RTL design principles and patterns
  \item Verilog coding best practices
  \item \textbf{Lab:} Design and synthesize HLS modules
\end{itemize}

\textbf{Afternoon Session (1:00 PM - 5:00 PM)}
\begin{itemize}
  \item Simulation-based verification
  \item Installing Icarus Verilog and GTKWave
  \item Writing effective testbenches
  \item Code coverage analysis with Covered
  \item \textbf{Lab:} Verify designs and analyze coverage
\end{itemize}

\subsubsection{Day 3: Logic Synthesis and Analysis}
\textbf{Morning Session (9:00 AM - 12:00 PM)}
\begin{itemize}
  \item Logic synthesis with Yosys
  \item Technology library setup (FreePDK45/SkyWater130)
  \item Synthesis optimization techniques
  \item Netlist generation and analysis
  \item \textbf{Lab:} Synthesize RTL designs to gates
\end{itemize}

\textbf{Afternoon Session (1:00 PM - 5:00 PM)}
\begin{itemize}
  \item Static timing analysis with OpenSTA
  \item Understanding timing constraints
  \item Setup and hold time verification
  \item Power analysis fundamentals
  \item \textbf{Lab:} Perform timing and power analysis
\end{itemize}

\subsubsection{Day 4: Physical Design with OpenROAD}
\textbf{Morning Session (9:00 AM - 12:00 PM)}
\begin{itemize}
  \item Installing and configuring OpenROAD
  \item PDK setup and technology files
  \item Floor planning concepts
  \item Standard cell placement strategies
  \item \textbf{Lab:} Floor plan and place a design
\end{itemize}

\textbf{Afternoon Session (1:00 PM - 5:00 PM)}
\begin{itemize}
  \item Clock tree synthesis (CTS)
  \item Global and detailed routing
  \item Timing-driven optimization
  \item Design rule compliance
  \item \textbf{Lab:} Complete physical implementation
\end{itemize}

\subsubsection{Day 5: Verification and Sign-off}
\textbf{Morning Session (9:00 AM - 12:00 PM)}
\begin{itemize}
  \item Physical verification overview
  \item DRC with Magic VLSI
  \item LVS verification with Netgen
  \item Parasitic extraction
  \item \textbf{Lab:} Run DRC/LVS on completed designs
\end{itemize}

\textbf{Afternoon Session (1:00 PM - 5:00 PM)}
\begin{itemize}
  \item GDSII generation and validation
  \item Layout visualization with KLayout
  \item Complete design challenge
  \item Presentation of results
  \item Workshop wrap-up and Q\&A
\end{itemize}

\subsection{Alternative Schedule (2-Day Intensive Format)}

\subsubsection{Day 1: Front-End Flow}
\begin{itemize}
  \item \textbf{8:00 - 9:00:} Registration and setup
  \item \textbf{9:00 - 10:30:} Introduction, toolchain, environment setup
  \item \textbf{10:30 - 12:00:} TCL scripting and automation
  \item \textbf{12:00 - 1:00:} Lunch break
  \item \textbf{1:00 - 3:00:} HLS, RTL design, verification
  \item \textbf{3:00 - 5:00:} Logic synthesis with Yosys
  \item \textbf{5:00 - 6:00:} Timing analysis with OpenSTA
\end{itemize}

\subsubsection{Day 2: Back-End Flow}
\begin{itemize}
  \item \textbf{8:00 - 9:00:} Day 1 recap and Q\&A
  \item \textbf{9:00 - 11:00:} OpenROAD installation and floor planning
  \item \textbf{11:00 - 12:30:} Placement and CTS
  \item \textbf{12:30 - 1:30:} Lunch break
  \item \textbf{1:30 - 3:30:} Routing and optimization
  \item \textbf{3:30 - 5:00:} DRC/LVS verification
  \item \textbf{5:00 - 6:00:} GDSII generation and final challenge
\end{itemize}

\subsection{Self-Paced Learning Path}
For independent study, follow this sequence:

\begin{enumerate}
  \item \textbf{Week 1:} Environment setup, Unix basics, TCL scripting
  \item \textbf{Week 2:} HLS and RTL design fundamentals
  \item \textbf{Week 3:} Verification and simulation techniques
  \item \textbf{Week 4:} Logic synthesis and timing analysis
  \item \textbf{Week 5:} OpenROAD installation and physical design basics
  \item \textbf{Week 6:} Advanced physical design and optimization
  \item \textbf{Week 7:} Physical verification (DRC/LVS)
  \item \textbf{Week 8:} Final project - complete RTL-to-GDSII flow
\end{enumerate}

\subsection{Required Materials}
Participants should have:
\begin{itemize}
  \item Laptop with minimum 8GB RAM (16GB recommended)
  \item Ubuntu 20.04+ or WSL2 installed
  \item At least 30GB free disk space
  \item Stable internet connection for tool downloads
  \item Text editor (VS Code, vim, or emacs recommended)
  \item PDF reader for workshop documentation
\end{itemize}

\subsection{Pre-Workshop Preparation}
To maximize workshop effectiveness:
\begin{enumerate}
  \item Install Ubuntu or WSL before arrival
  \item Review basic Unix commands
  \item Refresh Verilog knowledge (optional tutorials provided)
  \item Download workshop materials from repository
  \item Complete pre-workshop survey (if applicable)
\end{enumerate}

\subsection{Post-Workshop Resources}
After completion, participants will have access to:
\begin{itemize}
  \item Complete workshop documentation (PDF)
  \item All example designs and scripts
  \item Video recordings (if available)
  \item Online forum for continued support
  \item Reference designs and advanced tutorials
  \item Certificate of completion
\end{itemize}

\subsection{Support and Contact}
For workshop-related questions:
\begin{itemize}
  \item \textbf{Technical Support:} [Insert email]
  \item \textbf{Workshop Coordinator:} [Insert contact]
  \item \textbf{Online Forum:} [Insert URL]
  \item \textbf{GitHub Repository:} [Insert repository link]
\end{itemize}
