% == FILE: modules/module_power.tex ==
% Module: Power Analysis using OpenSTA

\subsection{Overview}
This module introduces \textbf{power analysis using OpenSTA}, an open-source Static Timing Analysis (STA) tool that also provides capabilities for estimating dynamic and leakage power. Continuing with our golden thread example, we analyze the power consumption of the \textbf{GCD unit} synthesized in Section 8.

\subsection{Requirements}
Before starting, ensure that:
\begin{itemize}
  \item \textbf{OpenSTA} is installed on your system. (Installation details are given in Tutorial 7.)
  \item You have the following files from previous sections:
    \begin{itemize}
      \item \texttt{gcd\_synth.v} – Gate-level netlist from Yosys (Section 8)
      \item \texttt{gcd.sdc} – Timing constraints file
      \item \texttt{FreePDK45\_osu\_stdcells.lib} – Technology library used in synthesis
    \end{itemize}
\end{itemize}

\subsection{Concepts of Power Dissipation}
In a CMOS circuit, total power dissipation has two major components:
\begin{enumerate}
  \item \textbf{Dynamic Power Dissipation} – due to switching activity.
  \item \textbf{Static (Leakage) Power Dissipation} – due to leakage currents in transistors.
\end{enumerate}

Dynamic power can be divided into:
\begin{itemize}
  \item \textbf{Internal Power:} Power dissipated within a cell due to charging/discharging of internal capacitances and short-circuit currents during switching.
  \item \textbf{Switching (External) Power:} Power consumed by charging/discharging external load capacitances driven by the cell.
\end{itemize}

\subsection{Internal and Switching Power Models}
Internal power is characterized in the technology library as a Non-Linear Power Model (NLPM).  
It is typically represented as a two-dimensional lookup table with:
\begin{itemize}
  \item Input transition time (slew) as one axis.
  \item Output capacitance (load) as the other axis.
\end{itemize}

Each entry in this table corresponds to the \emph{energy consumed per transition}, expressed in femtojoules (fJ).

\subsection{Files Used in this Tutorial}
\begin{itemize}
  \item \textbf{Verilog Netlist:} \texttt{gcd\_synth.v} – The synthesized GCD design containing FSM logic, datapath registers, subtractors, and comparators.
  \item \textbf{SDC File:} \texttt{gcd.sdc} – Defines timing constraints such as clock period, input/output delay, input slew, and output load.
  \item \textbf{Library:} \texttt{FreePDK45\_osu\_stdcells.lib} – Technology library defining power, delay, and leakage parameters for the standard cells.
  \item \textbf{TCL Script:} \texttt{gcd\_power.tcl} – Script to execute the power analysis commands.
\end{itemize}

\subsection{TCL Script Example}
Create a file \texttt{gcd\_power.tcl} with the following content:
\begin{lstlisting}[style=tclstyle]
# Read technology library
read_lib FreePDK45_osu_stdcells.lib

# Read the synthesized Verilog netlist
read_verilog gcd_synth.v

# Link instances to library cells
link_design gcd

# Read timing constraints
read_sdc gcd.sdc

# Set activity (signal switching probability)
# Higher activity for datapath due to subtractors
set_activity 0.2

# Report power
report_power
\end{lstlisting}

\subsection{Running the Power Analysis}
Run OpenSTA using the following commands:
\begin{lstlisting}[style=bashstyle]
sta
source gcd_power.tcl
\end{lstlisting}

This executes the TCL script, and the tool prints the power report including:
\begin{itemize}
  \item Internal power (per instance and total)
  \item Switching power (per net and total)
  \item Leakage power (per instance and total)
  \item Total power consumption
\end{itemize}

\subsection{Understanding GCD Power Characteristics}
The GCD design consumes more power than a simple inverter due to:

\textbf{Dynamic Power Contributors:}
\begin{itemize}
  \item \textbf{Subtractors:} Two 16-bit subtraction units (or one shared) with high switching activity during CALC state
  \item \textbf{Comparators:} Equality and greater-than comparison logic switching every cycle
  \item \textbf{Datapath Registers:} 16-bit registers for \texttt{a} and \texttt{b} updating frequently
  \item \textbf{FSM State Registers:} Transitioning between IDLE, CALC, and FINISH states
\end{itemize}

\textbf{Leakage Power Contributors:}
\begin{itemize}
  \item More standard cells = higher cumulative leakage
  \item Arithmetic cells (adders/subtractors) typically have higher leakage than simple logic gates
  \item All flip-flops contribute to leakage even when idle
\end{itemize}

\subsection{Power Breakdown Example}
For a typical GCD implementation with activity factor 0.2 and 100MHz clock:

\textbf{Internal Power:}
\begin{itemize}
  \item Each subtractor cell: $\sim$50-100 µW
  \item Flip-flops (state + datapath): $\sim$20-40 µW total
  \item Combinational logic: $\sim$30-50 µW
  \item \textbf{Total Internal:} $\sim$100-200 µW
\end{itemize}

\textbf{Switching Power:}
\begin{itemize}
  \item Depends heavily on net capacitance and toggle rate
  \item Datapath nets (16-bit buses) contribute significantly
  \item \textbf{Total Switching:} $\sim$50-100 µW
\end{itemize}

\textbf{Leakage Power:}
\begin{itemize}
  \item FreePDK45 has moderate leakage characteristics
  \item $\sim$50-100 cells in the design
  \item \textbf{Total Leakage:} $\sim$10-20 µW
\end{itemize}

\textbf{Total Power:} $\sim$160-320 µW (varies with activity and technology corner)

\subsection{Impact of Optimization}
If you synthesized with resource sharing (Section 8), compare power:
\begin{itemize}
  \item \textbf{Unoptimized (two subtractors):} Higher dynamic power due to more switching, higher leakage due to more cells
  \item \textbf{Optimized (one subtractor):} Lower dynamic and leakage power, but added mux power
  \item \textbf{Net result:} Optimized version typically saves 15-25\% total power
\end{itemize}

\subsection{Interpretation of Power Report}
The OpenSTA report will show:
\begin{itemize}
  \item \textbf{Per-instance breakdown:} Identify which modules consume the most power (usually subtractors)
  \item \textbf{Per-net switching:} High-fanout nets (like clock and reset) contribute significantly
  \item \textbf{Hierarchical view:} If your design has hierarchy, see power per sub-module
\end{itemize}

Look for:
\begin{itemize}
  \item Unexpectedly high power in specific instances (may indicate over-sized cells)
  \item High switching power on specific nets (may need buffering or optimization)
  \item Leakage dominated designs (may benefit from power gating or lower-leakage cells)
\end{itemize}

\subsection{Exercise}
\begin{enumerate}
  \item Run power analysis on both optimized and unoptimized GCD netlists. Compare the results.
  \item Modify the activity factor in the TCL script (try 0.1, 0.3, 0.5) and observe impact on dynamic power.
  \item Change the clock frequency in the SDC file and re-run analysis. Plot power vs. frequency.
  \item Identify the top 5 power-consuming instances in the design. What cell types are they?
  \item \textbf{Challenge:} If you have the HLS-generated netlist, compare its power to your manual RTL version.
\end{enumerate}

\subsection{Summary}
This tutorial demonstrated power analysis of the GCD unit using OpenSTA. You learned:
\begin{itemize}
  \item How to analyze synthesized netlists for power consumption
  \item The breakdown of internal, switching, and leakage power
  \item Why arithmetic-heavy designs like GCD have higher power than simple logic
  \item The impact of optimization techniques on power consumption
\end{itemize}

Understanding power consumption early in the design flow helps make informed decisions about optimization, clock frequency, and target technology before investing in physical implementation.
